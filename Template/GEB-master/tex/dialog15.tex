
\begin{dialog}{生日大合唱哇哇哇乌阿乌阿乌阿……}

\begin{quote}
五月里的一个晴朗日子,乌龟遇见了阿基里斯,两人便一起漫步在树林里。阿基里斯穿得漂漂亮亮,摇头晃脑地哼着一段曲子,别在他衬衣上的一枚硕大的徽章上写着:“今天是我生日!”
\end{quote}

\begin{dialogue}

\item[乌龟]你好啊,阿基,什么事让你这么高兴?是不是赶上你生日了?

\item[阿基里斯]对,对!是的,今天是我的生日!

\item[乌龟]我猜也是。你那徽章上写着呢。而且,要是我没听错的话,你唱的是巴赫的《生日大合唱》。那是他1727年为萨克森国王奥古斯都五十七岁生日而写的。

\item[阿基里斯]你说的对。奥古斯都的生日和我一样,所以这首《生日大合唱》就有双重意义。不过我不想告诉你我的年龄。

\item[乌龟]喔,这完全在你。可我想知道另外一件事,从你已经告诉我的这些情况里,就能得出结论说今天是你的生日,对吗?

\item[阿基里斯]对,对!完全正确。今天是我的生日。

\item[乌龟]好极了。这正和我猜的一样。我将要推定今天是你的生日,除非——

\item[阿基里斯]——除非什么?

\item[乌龟]除非那是一个轻率仓促给出的推断。我们乌龟毕竟不喜欢跳跃推理(我们压根就不喜欢跳,尤其不喜欢跳跃推理。)我深知你也喜欢逻辑思维,那么请问,根据前面那些话,从逻辑上演绎出今天是你的生日,合理不合理?

\item[阿基里斯]我相信我已经看出你提问题的规律了,龟兄。但我这回不跳着推理了,我要按字面意思对待你的问题,直截了当地回答你:是。

\item[乌龟]好!好!那就只有一件事了,我需要知道它以便完全确定今天是——

\item[阿基里斯]对,对,对,对……我明白你要问些什么,龟兄。我得让你知道,我已不像不久前讨论欧几里得证明时那么容易上当了。

\item[乌龟]哪儿的话呀,谁说过你容易上当啦?正相反,我倒认为你是逻辑思维方面的专家、演绎科学的权威、正确推理方法的泰斗……说真的,阿基,依我看,你是推理艺术的巨匠。正因为如此,我要问你:“前面那些话是否提供了足够的证据,使我不用再费脑筋就能推出今天是你的生日?”

\item[阿基里斯]你对我真是过重了,龟兄,噢,我是说——过奖了!不过我注意到了你在翻来覆去地提类似的问题——要我说,你(就跟我一样)每次都能回答“是”。

\item[乌龟]我当然能,阿基。可要是那样做就得靠运气了,而我们乌龟——不像那些瞎猫——讨厌撞见死耗子这种事,乌龟只肯作合理猜测。啊,是的——合理猜测的威力。你想象不出有多少人在作猜测的时候,没有把全部的相关因素都考虑进去。

\item[阿基里斯]依我看,我觉得这一大堆废话里只有一个相关因素,那就是我的头一句话。

\item[乌龟]嗯,的确,我承认它至少是一个该考虑的因素——不过,你不会要我无视逻辑吧?那可是一门受人尊敬的学问,作合理猜测时,逻辑永远是一个相关因素。现在,既然有一位著名的逻辑学家和我在一起,我在思考时就只按符合逻辑的方式借助刚才那个因素,并且要通过直接问你我的直觉是否正确来证实我的预感。因此,我最后还是得直截了当地问问你:“上面那些话是否可以让我丝毫不差地推出今天是你的生日?”

\item[阿基里斯]我再说一遍:是。不过坦率地说,我明确地感到你自己已经可以给出答案了,就像前面几次一样。

\item[乌龟]你也太刻薄了!我倒真想能如你所说的那么聪明呢!可我只不过是只普普通通、极为无知的乌龟,渴望考虑到全部的相关因素,所以我需要所有这些问题的答案。

\item[阿基里斯]那好,我就来一劳永逸地解决这个问题。你先前所提的全部问题,以及你今后按这种方式将提出的全部问题,答案都一样,那就是:是。

\item[乌龟]好极了!一网打尽。你用自己独创的方法,避免了所有麻烦。我把这个巧妙的发明叫做答案模式,你不会见怪吧?答案模式把第一个“是”答案、第二个“是”答案、第三个“是”答案等等,统统缠在一起。事实上,当它缠下去一直缠到头时,它该有个名称:“$\omega$答案模式”,“$\omega$”是最末一个希腊字母——我不必给你讲这个了吧?

\item[阿基里斯]我不在乎你叫它什么。现在我总算轻松了,你终于承认了今天是我的生日。现在我们可以谈点别的了——比如说,你打算送我什么礼物?

\item[乌龟]等会儿——别这么急。我会承认今天是你的生日,只是要有一个条件。

\item[阿基里斯]什么条件?是我不要礼物吗?

\item[乌龟]不是不是。阿基,我其实正打算招待你一顿丰盛的生日晚餐——只要你让我确信,一下子给出全部那些“是”答案(由$\omega$答案模式给出的),我就能不绕弯子而直接推出今天是你的生日。是这么回事吧?

\item[阿基里斯]对,就是这么回事。

\item[乌龟]好,那我现在就有第$\omega+1$个“是”答案了。有了它以后,我就可以进一步来试着接受今天是你生日这个假设——如果上面那种做法有效的话。你是不是能在这件事上给我当当顾问,阿基?

\item[阿基里斯]都是些什么呀?我已经看穿了你那个无穷把戏。你不是不满足于第$\omega+1$个“是”答案吗?那好,我就不光给你第$\omega+2$个“是”答案,而且还给你第$\omega+3$个,第$\omega+4$个,一直给下去。

\item[乌龟]你真大方,阿基。今天是你的生日,该我送你礼物而不是反过来让你送我。直说了吧,我怀疑今天是不是你的生日。现在,有了这个新的答案模式(我要叫它$2\omega$答案模式),我想我可以推出今天是你的生日了。不过请你告诉我,阿基,$2\omega$答案模式确实能使我完成这一飞跃吗?我是不是又漏掉了什么东西?

\item[阿基里斯]你用不着再来套我。我已经有办法结束这场无聊的游戏了。我给你一个结束全部答案模式的答案模式!也就是说,我同时给你一系列答案模式:$\omega$,$2\omega$,$3\omega$,$4\omega$,$5\omega$,一直下去。用这个元答案模式,我跳出整个系统,一了百了,超越了你自以为套住了我的这个无聊游戏——现在完事大吉了!

\item[乌龟]嗬!阿基,我很荣幸能接受这样一个强有力的答案模式。我觉得它一定是很少见的,人的头脑竟能发明出如此庞大的东西,我看着都害怕。我来为你这一馈赠起个名字吧,你看如何?

\item[阿基里斯]随你便。

\item[乌龟]那我就叫它“$\omega^2$答案模式”。我们马上就可以干别的了——不过你要告诉我,有了$\omega^2$答案模式,我是否就能推导出今天是你的生日?

\item[阿基里斯]唉,真倒霉!这一连串恼人的问题还有没有个完?下一个该什么啦?

\item[乌龟]哦,$\omega^2$答案模式之后,还有第$\omega^2+1$个答案。然后还有第$\omega^2+2$个,一直下去。不过你又可以把它们全打成一捆,作为$\omega^2+\omega$答案模式,然后就会有其它的捆,诸如$\omega^2+2\omega$、$\omega^2+3\omega$……最后,能到达$2\omega^2$答案模式,这后面,又有$3\omega^2$答案模式和$4\omega^2$答案模式。它们后面,还有更进一步的答案模式,诸如$\omega^3$、$\omega^4$、$\omega^5$等等。就是这么进行下去,颇有些共同风格呢。

\item[阿基里斯]我能想象了。我看,过一会就该到$\omega^\omega$答案模式了。

\item[乌龟]那当然。

\item[阿基里斯]然后是$\omega^{\omega^\omega}$和$\omega^{\omega^{\omega^\omega}}$对吗?

\item[乌龟]你领会的快极了,阿基。要是你不介意,我想提个建议:你干嘛不把这些答案模式全都塞进单独一个答案模式里去呢?

\item[阿基里斯]好吧,我真怀疑这样做能有什么结果。

\item[乌龟]依我看,我们至今所提出的命名惯例显然都不适于给它命名。我们随便起个名字。叫它$\varepsilon_0$答案模式吧。

\item[阿基里斯]真讨厌!每次都给我的答案起个名字。我本希望这个答案能使你满意,可这么一来我的希望立刻又落空了。我们刚才干嘛不留着这个答案模式不起名字呢?

\item[乌龟]哪有那么容易!不起名字就没法称呼它了。此外,这个特别的答案模式是有点不可避免的,而且还相当漂亮,要是连名字都没有就太不雅观了!你过生日时总不想做些什么观之不雅的事情,对吗?或者,今天不是你的生日吧?要说生日,今天该是我的生日!

\item[阿基里斯]什么?今天是你的生日?

\item[乌龟]对,没错。不过说老实话,今天是我叔叔的生日,但那差不多是一回事。今天晚上你打算怎么请我吃美味可口的生日晚餐啊?

\item[阿基里斯]得了吧,老龟!今天是我的生日,该你作东!

\item[乌龟]啊,可你一直没说服我相信你这个说法是真的。你热衷于用那些答案、答案模式等等东拉西扯,可我想知道的是:今天是不是你的生日。你整个把我搞晕了。唉,太不幸了。不过,无论如何,你今晚要是请我吃一顿生日晚餐,我将十分高兴。

\item[阿基里斯]那好吧,我正好知道一个地方,那里有各种各样可口的汤,而我也恰好知道我们该吃哪一种……

\end{dialogue}

\end{dialog}
