
\begin{dialog}{二部创意曲}

\removelastskip
\bigskip

\begin{center}
或\\[\bigskipamount]\nointerlineskip
{\kaishu\large 乌龟说给阿基里斯的话}\\[\medskipamount]
刘易斯·卡罗尔\lnote{作}\note{刘易斯·卡罗尔,“乌龟说给阿基里斯的话”,《心智》[Mind],新号第2期(1895年),第278--280页。}
\end{center}

\bigskip

\begin{quote}
阿基里斯追上乌龟之后,舒舒服服地坐在了龟背上。
\end{quote}

“这么说,你已经到达我们这场赛跑的终点了,是吗?”乌龟说。“虽然比赛的路程是由无数段路程组成的,你还是跑到啦?我记得有个自作聪明的人证明过这是不可能做到的。”

“这是可以做到的,”阿基里斯说。“这已经做到了!马到成功。你瞧,那些路程在不断缩短,因此——”

“可要是它们一直在不断增长呢?”乌龟打断他的话说。“那会怎么样?”

“那样一来我现在就不会在这儿了,”阿基里斯老老实实地说。“而你到这时应该已经绕地球好几圈了!”

“你对我真是过奖了——哦,我是说你真是过重了,”乌龟说。“因为你的确太沉了!哎,你想听听有关另一场比赛的事吗?在这场比赛里大多数人都以为他们在两三步之内就能达到终点,实际上这场比赛也是由无数段路程组成的,其中每一段都比它前面那段要长些。”

“太想知道了!”那位希腊勇士一边说,一边从他的头盔里(那时几乎没有哪个希腊勇士身上有口袋)拿出一个硕大的笔记本和一支铅笔。“开始吧!请慢点说!速记现在还没发明出来呢!”“那个漂亮的欧几里德第一命题!”完全沉浸进去的乌龟喃喃地说。“你钦佩欧几里德吗?”

“钦佩得五体投地!至少,对于一部未来几个世纪之后才将问世的著作来说,我钦佩它的程度是无与伦比的了!”

“那好,让我们先来看看第一命题的论证中一个小小的片断——仅仅两步——以及由此得出的结论。请把它们记在你的本上。为了谈到它们时方便起见,我们将它们称作A、B、Z:——
\begin{description}
\item[(A)] 同等于一物的彼此亦相等。
\item[(B)] 这个三角形的两条边同等于一物。
\item[(Z)] 这个三角形的两条边彼此相等。
\end{description}
我想欧几里德的读者们会认为Z是A和B的合乎逻辑的推论,所以任何人只要认为A和B为真,则必定认为Z也真,不是吗?”

“毫无疑问!连上初中的毛孩子——等到两千多年以后发明出初中来——也会对此表示同意的。”

“如果有某个读者不认为A和B为真,我想他也还会认为这一推论是有效的,对吗?”

“这种读者肯定有。他会说,‘我同意下述假言命题是真的:如果A和B为真,则Z必为真。但是我不接受A和B是真的。’这种读者应该明智点儿,放弃欧几里德,改踢足球去。”

“还可能有读者会说‘我同意A和B为真,但不接受那个假言命题’,这也有可能吧?”

“当然有这种可能。他最好也去踢足球。”

“迄今为止,”乌龟继续说,“这两种读者都没有接受Z为真的逻辑必然性,对吗?”

“没错,”阿基里斯赞同地说。

“嗯,那么,我想让你把我当作是那第二种读者,用逻辑来迫使我接受Z为真。”

“一只会踢足球的乌龟会是——”阿基里斯有点吞吞吐吐。

“——当然是挺反常的,”乌龟急忙抢过话头。“你先别往旁边岔。咱们先谈Z,后谈球!”

“我得迫使你接受Z,是吗?”阿基里斯若有所思地说。“你现在的观点是只接受A和B,而不接受假言——”

“把它称作C吧,”乌龟说。

“——而不接受
\begin{description}
\item[(C)] 如果A和B为真,Z必为真。”
\end{description}

“这就是我目前的观点,”乌龟说。

“那么我得说服你必须要接受C。”

“只要你一把它记在你的本子上,我马上就接受它。”乌龟说,“你在上面还记了些什么?”

“只是一些摘要,”阿基里斯一边说,一边紧张地翻着那个笔记本:“是些摘要——是关于使我出了名的那些场战斗的!”

“我看到有好多空白页!”乌龟叫道,这一发现使他很高兴。“它们都会用得上的!”\dlnote{(阿基里斯颤了一下。)}“现在,我说你写:——
\begin{description}
\item[(A)] 同等于一物的彼此亦相等。
\item[(B)] 这个三角形的两条边同等于一物。
\item[(C)] 如果A和B为真,则Z必为真。
\item[(Z)] 这个三角形的两条边彼此相等。”
\end{description}

“你应该把最后这个命题Z改称为D,”阿基里斯说。“它紧承着上面那三个命题而来。如果你接受A、B、C,你就必须接受Z。”

“为什么我必须接受?”

“因为它是前三个的合乎逻辑的推论。如果A、B、C为真,则Z必为真。我看这是无可争辩的吧?”

“如果A、B、C为真,则Z必为真,”乌龟若有所思地重复着。“这又是一个假言判断,对不对?如果我不觉得这一假言判断是真的,那么我可以接受A、B、C,而仍然不接受Z,对吗?”

“你可以,”这位正直的英雄承认道。“可头脑这么迟钝的人也太罕见了。不过,这事儿还是可能的。所以我必须再使你接受一个假言判断。”

“好极了,我很愿意接受,只要你把它写下来。我们可把它称作
\begin{description}
\item[(D)] 如果A和B和C为真,则Z必为真。
\end{description}
你把它记到你的本上了吗?”

“已经记上了!”阿基里斯快活地宣布,同时把铅笔插进笔帽里。“我们终于达到了这场思想竞赛的终点!你现在接受了A、B、C、D,你当然要接受Z。”

“是吗?”乌龟装傻充楞地说。“让我们搞搞清楚。我接受了A、B、C、D,请设想一下,我依然拒绝接受Z,那会怎么样?”

“那样逻辑就会掐着你的脖子,迫使你接受!”阿基里斯得意洋洋地回答道。“逻辑会告诉你,‘这事你作不了主。你既然已经接受了A、B、C、D,你就必须接受Z!’因此你别无选择,明白了吗?”

“这美妙的逻辑提供给我们的一切,都值得记下来,”乌龟说。“请把它记在你的本上吧。我们将把它称作
\begin{description}
\item[(E)] 如果A、B、C、D为真,则Z必为真。
\end{description}
在我接受它之前,我当然不一定要接受Z。因此它是很必要的一步,你明白吗?”

“我明白,”阿基里斯说。他的声音里带着点悲哀。

到这里,叙述者由于家里有急事,不得不离开这可爱的一对儿了。直到几个月以后,他才又路过这个地方。这时阿基里斯还坐在耐心的乌龟的背壳上,往他那几乎写满了的本子上写个不停。乌龟这时说:“你把刚才那步记下了吗?我要是没数错,已经有一千零一步了。将来还会有亿万步呢。算是帮我个忙,你不会介意考虑考虑我们两人的这篇对话究竟给了十九世纪的逻辑学家多少教益吧——你不会介意利用一下我表兄假海龟到那时将会发明出来的那个双关语,把你的名字改成‘悟诡’吧?”

“随你便,”这位精疲力竭的勇士把脸埋在双手里回答说,他的声音因绝望而变得很空洞。“只要你愿意利用假海龟从未发明过的双关语,把你的名字改成‘厌极易死’!”

\end{dialog}
