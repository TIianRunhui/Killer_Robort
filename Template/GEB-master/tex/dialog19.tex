
\begin{dialog}{对实}

\begin{quote}
螃蟹邀请了几个朋友来看星期六下午电视中播放的橄榄球赛。阿基里斯已经到了,可还要等等乌龟和他的朋友树懒。
\end{quote}

\begin{dialogue}

\item[阿基里斯]驭骑在那辆与众不同的独轮车上的两个人,会是我们那两位朋友吗?

\dnote{(树懒和乌龟下了车走进来)}

\item[螃蟹]哦,朋友们,你们光临我真高兴。请让我介绍一下我最爱戴的老朋友,树懒先生——这是阿基里斯。我敢肯定你认识乌龟。

\item[树懒]就我记忆所及,这是我有生以来第一次结识一位双目独眼龙。很高兴见到你,阿基里斯。关于双目独眼类的生物,我已经与闻不少了。

\item[阿基里斯]我也是。我可以了解了解你那别致的交通工具吗?

\item[乌龟]你是问我们那辆双轴独轮车?谈不上什么别致。只不过是供两个人以同一速度从A到B的一种交通工具。

\item[树懒]它是由生产“双头翘”翘翘板的公司制造的。

\item[阿基里斯]噢,原来是这样。车上那个球形把手是什么东西?

\item[树懒]那是调挡装置。

\item[阿基里斯]哦!那它有多少种速度?

\item[乌龟]一种,包括回动。许多别的型号的比这更少,可这辆是特型的。

\item[阿基里斯]一看就是辆挺不错的双轴独轮车。噢,老蟹,我早就要告诉你,昨晚你的管弦乐队的演出我非常喜欢。

\item[螃蟹]谢谢,阿基你当时也在吗?小懒?

\item[树懒]不,我不巧没能到场。我那时正参加一场混合单打乒乓球锦标赛。由于我所在的队陷入了无对手的争夺冠军的决赛,所以特别扣人心弦。

\item[阿基里斯]你获得什么奖品了吗?

\item[树懒]当然——是一条一面镀金、一面镀银的铜制莫比乌斯带。

\item[螃蟹]祝贺你,小懒。

\item[树懒]谢谢,哎,跟我讲讲那个音乐会吧。

\item[螃蟹]那是场最美妙的演出。我们演奏了一些巴赫孪生兄弟的作品——

\item[树懒]是著名的约翰和塞巴斯第安吗?

\item[螃蟹]反正都是一回事。有一部作品让我想到了你,小懒——那是一首用两只左手弹的了不起的钢琴协奏曲。其中倒数第二个乐章(也是仅有的一个乐章)是一支单声部的赋格。你简直没法儿想象它有多复杂。我们还演奏了贝多芬的第九交响曲作为压轴节目。演出结束时,听众们全体起立,每个人都用一只手鼓掌,真是掌声雷动啊。

\item[树懒]哦,我没看上真是太可惜了。不过,不知你录了音没有?我家里有一台高保真的录音机——这是你所能买到的最好的双声道单喇叭的音响系统。

\item[螃蟹]我敢肯定你能搞到那录音。好啦,朋友们,球赛就要开始啦。

\item[阿基里斯]今天谁跟谁比赛,老蟹?

\item[螃蟹]一定是主队对客队。哦。不——那是上周。我想这周是本地队对外地队。

\item[阿基里斯]我为本地队加油。我一向这么做。

\item[树懒]啊,那你可太地方主义了。我从不为本地队加油。越是远在十万八千里的队,我越是要给他们加油。

\item[阿基里斯]噢,那么你是住在十万八千里啦?我听说那儿很迷人,不过我还是不想去那儿。毕竟太遥远了。

\item[树懒]而且奇怪的是,你无论走哪条路,都不会更近些。

\item[乌龟]那是种对我路子的地方。

\item[螃蟹]球赛就要开始了。我看得打开电视了。

\dnote{(他走到一个有屏幕的巨大的箱子前,箱子下面有个仪表盘,复杂得跟喷气式飞机上的一样。他轻轻按了一下一个球形开关,屏幕上出现了色彩鲜亮、图像清晰的橄榄球场。)}

\item[解说员]下午好,球迷们。一年一度的本地队和外地队在绿茵场上的较量又拉开战幕了。今天下午断断续续地下了点小雨,场地多少有点湿。不过,尽管天公不作美,这场比赛肯定会非常精彩,特别是本地队有两名著名的八分位,他们是提里趿拉和回文斯。大家知道,根据规则,美式足球——也就是这种橄榄球——每个球队里应该只有一名四分位指挥进攻,不过只要比赛精彩,弄成两名八分位也没有关系,不是吗?

\item[乌龟]天哪!这种比赛我还从没见识过。

\item[解说员]现在丕里帕拉为本地队开球。球飞在空中!外地队的法兰孙接到了皮球,他抱球往回跑——跑到了$20$码线的地方,到了$25$码线,到$30$码线,他在$32$码的地方摔倒了。是本地队的哞呕一个擒抱,把他摔倒的。

\item[螃蟹]好漂亮的反攻!你们看到没有,他差一点叫悟诡给擒抱住——可他还是挣脱掉了。

\item[乌龟]叫谁擒抱?

\item[螃蟹]悟诡!

\item[乌龟]怎么跟我的名字就差一点儿?

\item[树懒]别犯傻了,老蟹。没这么回事,悟诡没有擒抱法兰孙。没有必要用这种“差一点”发生的事儿蒙可怜的阿基里斯(和我们别人)。事实就是事实——没有什么“差一点”啦、“要是”啦、“还”啦、“可是”啦这些东西。

\item[解说员]这是重放的镜头。注意$79$号队员悟诡,他从边路包抄过来,威胁法兰孙,打算擒抱他!

\item[树懒]“打算”!哼!

\item[阿基里斯]这一招儿可太漂亮啦!要是没有重放镜头我们怎么能看得到!

\item[解说员]外地队第一次进攻,攻进十码。现在球在闹得乐手中,他把皮球传给了牛韦——这是个后传球——牛韦绕到右面,把球掷给了法兰孙——这是个双后传球,诸位!——现在法兰孙把球递给华果树,华在争球线后面十二码的地方达阵得分。这个三后传球丢了十二码!

\item[树懒]太棒了!真是一场扣人心弦的比赛!

\item[阿基里斯]哦,小懒,我还以为你一直在为外地队加油呢。这场球他们丢了十二码。

\item[树懒]他们丢了十二码?啊——只要他们表演得出色,管它输还是贏呢!我们再看一遍吧。

\dnote{(……就这样,上半场比赛结束了。在第三节比赛就要结束的时候,本地队决定性的时刻来到了。这时他们落后八分。这是他们第三次进攻,还一码未进,他们这时急需继续进攻的机会)}

\item[解说员]球被抛到提里趿拉手里,他退后几步,找人传球,晃过了悟诡。这时回文斯在紧右边,附近没有别的队员。提里趿拉看到了他,来了一个低传,把球传到回文斯手里。回文斯接住球以后,就——\dlnote{(可以听到人群中发出的惋惜声。)}——唉,他跨出了边线!各位观众,这对本地队可是个沉重打击。要是回文斯没有跨出边线,他会一直跑到底线区,达阵得分的。现在让我们看一看虚拟的重放镜头。

\dnote{(屏幕上出现了同刚才一样的阵容。)}

球被抛到提里趿拉手里,他退后几步,找人传球,晃过了悟诡。这时回文斯在紧右边,附近没有别的队员。提里趿拉看到了他,来了一个低传,把球传到回文斯手里。回文斯接住球以后,就——\dlnote{(可以听到人群中发出的紧张的惊叹声。)}——他差一点跨出边线!不过他仍在界内,通向得分区的路上完全没有对方队员。回文斯一往直前,达阵得分!\dlnote{(球场内欢声雷动。)}啊,球迷们,要是回文斯没有跨出边线,这就是将会发生的情形。

\item[阿基里斯]等等……达阵得分这事有还是没有?

\item[螃蟹]哦,没有。这只是虚拟的重放镜头。这只不过是演示了一下假设而已。

\item[树懒]这是我听到过的最荒唐的事。我看他们还会发明水泥护耳皮套呢。

\item[乌龟]虚拟的重放镜头有点少见,是吧?

\item[螃蟹]要是你有一台虚拟电视,也没有什么特别的。

\item[阿基里斯]需你电视?是一种非得你亲自操作不可的电视吗?

\item[螃蟹]不,你理解岔了。这是种可以进入虚拟状态的新型电视,特别适用于观看球赛之类的节目。这是我刚搞到的。

\item[阿基里斯]它上面为什么有这么多的球形开关和叫人眼花缭乱的仪表盘?

\item[螃蟹]你可以用它们来调你所需要的频道。有很多频道都在广播虚拟的节目,你能很方便地从中选择。

\item[阿基里斯]你能把你说的表演给我们看看吗?恐怕我还不太明白“广播虚拟的节目”是什么意思。

\item[螃蟹]哦,其实很简单。你自己也能弄明白。我现在去厨房做点油炸土豆片,我知道小懒很爱吃。

\item[树懒]噢!去吧,老蟹!油炸土豆片是最对我口味的食品。

\item[螃蟹]你们别的人呢?

\item[乌龟]我也想吃点。

\item[阿基里斯]我也想。可等等——在你去厨房之前,能吿诉我使用你的虚拟电视有什么窍门吗?

\item[螃蟹]没什么特别之处,尽管接着往下看,什么时候有某种几乎要成却没成的事,什么时候你希望某件事情不是这样发生,而是换一个样子,你就摆弄摆弄那些开关,看看会发生什么事。你也许会调出些稀奇古怪的频道来,但你不会把它们弄坏的。

\dnote{(他进了厨房)}

\item[阿基里斯]我不明白他是什么意思。嗯,好吧,我们还是回到这场比赛上来吧,我真叫它迷住了。

\item[解说员]这是外地队第四次进攻,本地队为守方。外地队采取踢悬空球的战术,提里趿拉位置靠近得分线。牛韦将球踢回——这是个高飘球,皮球朝着提里趿拉落下去——

\item[阿基里斯]接球,提里趿拉!给外地队一点颜色看看。

\item[解说员]——球落到了水坑里——噗哧!球反弹的方向很奇怪!斯普鲁克猛地一个飞身接球!反弹起的皮球几乎擦着提里趿拉的身边滑开了——这个球被判为失球。裁判判凶猛的斯普鲁克为外地队得分,本地队落后七码。这对本地队是很不利的。这只能说是老天不想成全他们。

\item[阿基里斯]噢,不!要是没有这场雨……\dlnote{(绝望地绞着双手。)}

\item[树懒]又一个该死的假设!你们怎么都陷进自己幻想的荒诞世界里去了!我要是你们,我就不会违背现实。我的格言是“避免虚拟的谵语”。即使别人给我一百块——不,一百一十二块——油炸土豆片,我也不会放弃我的格言。

\item[阿基里斯]哎,这倒提醒了我。没准儿摆弄摆弄这些球形开关,就能碰巧找到不下雨的虚拟重放镜头,那样一来就没有泥水坑了,皮球也不会向稀奇古怪的方向弹了,提里趿拉也就不会失球。我想知道……\dlnote{(走到虚拟电视机前,盯着它瞧了一会儿。)}可我一点也不知道这些球形开关都是管什么的。\dlnote{(随便拧了几下。)}

\item[解说员]这是外地队第四次进攻,本地队为守方。外地队采取踢悬空球的战术,提里趿拉位置靠近得分线。牛韦将球踢回——这是个高飘球,皮球朝着提里趿拉落下去——

\item[阿基里斯]接球,提里趿拉!给外地队一点颜色看看。

\item[解说员]球落到了水坑里——噗哧!球反弹到他手里。斯普鲁克猛地从他身后飞起抢球,却被他有效地阻挡住了,他避开凶猛的斯普鲁京,在他前面是一片无人之境。注意,他到了$50$码,到了$40$码,$30$码了,到$20$码,$10$码——达阵得分,本地队得分!\dlnote{(本地队一方欢声如雷)}球迷们,要是橄榄球不是橄榄形的,而是西瓜形的,这就是将会出现的情况!不过,在现实中,本地队失了球,而外地队攻到了本地队的七码线上。这真是皮球无知,不解人心。

\item[阿基里斯]这你怎么想,小懒?

\dnote{(阿基斯冲着树懒得意洋洋地笑了一声,可后者却完全没有注意,他正全神贯注地瞧着螃蟹端着一只大盘子走了进来,那盘子上盛着一百一十二块——不——一百块又大又馋人的油炸土豆片,以及为每个人准备的餐巾。)}

\item[螃蟹]你们觉得我的虚拟电视怎么样?

\item[树懒]坦率地说,老蟹,真叫人失望。它好像出了毛病。至少有一半时间里它是在瞎捣乱。这要是我的,我会马上不要了,把它给一个像你这样的人,不过,这当然不是我的电视。

\item[阿基里斯]这是件很古怪的玩艺儿。我想让它重演在另一种天气下的比赛,可这玩艺儿似乎自有一套想法!它不去改变天气,而是把橄榄球变得不是橄榄形的,却是西瓜形的!那么请告诉我——橄榄球怎么能不是橄榄形的?这真是自相矛盾。太荒诞了!

\item[螃蟹]真没劲!我还以为你们会发现什么有趣的虚拟呢。要是把橄榄球赛改成棒球赛,刚才那场比赛的结局会怎么样,你们想看看吗?

\item[乌龟]噢,这主意太伟大啦!

\dnote{(螃蟹拧了拧两个球形开关,然后退了几步。)}

\item[解说员]现在有四个人出局,还——

\item[阿基里斯]四人出局?

\item[解说员]对,球迷们——四人出局。你一旦把橄榄球赛改成棒球赛,就得接受一些既成事实!我还是要说,四人出局,外地队是守队,本地队进攻。现在提里趿拉打垒。外地队采取触击战术。牛韦现在举手投球——他投了一个上升球。球朝着提里趿拉直飞过来——

\item[阿基里斯]狠扣它一个,提里趿拉!给外地队一点颜色看看!

\item[解说员]——但是球好像沾了水,在空中划了一道奇怪的弧线。斯普鲁克猛地飞身抱球!那球看去险些擦过提里趿拉的球棒,从他身边弹开了——这被判为飞球。裁判判凶猛的斯普鲁克为外地队得分,从而结束了第七局的比赛。这对本地队很不利。要是把橄榄球赛改成棒球赛,这就是刚才的比赛将出现的结局。

\item[树懒]哼!你们干脆把比赛挪到月球上去算啦。

\item[螃蟹]说干就干!只要在这儿拧一下,在那儿拧一下……

\dnote{(屏幕上出现了一块孤零零的火山口样的场地,双方队员都穿着太空服定在原地不动。但很快,两队队员同时动起来,队员们跳得很高,有时都高过了其他队员的头顶。球被抛向空中,高到几乎看不见了。然后再缓慢地落到某个身穿太空服的队员手里,球落下的地方离它被抛起的地方几乎有四分之一公里远)}

\item[解说员]朋友们,你们现在看到的是虚拟的比赛在月球上进行时的情景。现在让我们先抽空儿告诉大家一点重要的商业行情。提供这一信息的好心人专酿古鲁皮啤酒——这是最对我口味的啤酒——待会儿我们再回到球赛上去。

\item[树懒]要是我不这么懒的话,我一定要亲自把这台破电视奉还给卖主!可是,唉,我已经命中注定要做一个懒惰的树懒了……

\dnote{(嚼了一大口油炸土豆片。)}

\item[乌龟]嘿,这可是个了不起的设想,老蟹。我可以提出一种假设情况吗?

\item[螃蟹]当然可以。

\item[乌龟]要是空间变成四维的,刚才那场比赛会怎么样?

\item[螃蟹]哦,这可够复杂的,龟兄,不过我看我还是能够把这种构想输进仪器里面。请稍候。

\dnote{(他走上前去,看来是第一次充分使用了那台虚拟电视机的控制盘,几乎每只球形开关都被拧了两三次。他仔细地检查了各种控制器的数据,然后退回来,脸上带着一副满意的神情。)}

我看这下行了。

\item[解说员]现在让我们来看看虚拟的重放镜头。

\dnote{(屏幕上出现了许多混乱地绞在一起的管。一会儿变大,一会儿变小,有时又似乎在旋转。随后又变成某种奇特的蘑菇状的东西,到最后又回到管的形状。就在那些图像一会儿这样、一会儿那样,变成各种稀奇古怪的图形时,解说员又开始解说了。)}

提里趿拉退后几步。他看到了场外十码处的回文斯,把球向右向外抛给了回文斯——看来很成功!回文斯现在到了$35$码平面,$40$码,他在场地上自己一方$43$码的平面上被擒抱了,三维的球迷们,要是球赛在四维空间里举行,这就是将会发生的情况。

\item[阿基里斯]你对它做了些什么,老蟹,你是怎么摆弄控制盘上那些各式各样的开关的?

\item[螃蟹]我选择了适当的虚拟频道。喏,有这么多的虚拟频道同时广播,我只是把它调到广播我们所要求的那种虚拟的频道上。

\item[阿基里斯]随便什么电视机都能这么来吗?

\item[螃蟹]不,大多数电视机无法接收虚拟频道。这需要一套特殊的、极难制作的电路。

\item[树懒]你怎么知道哪一个频道播放哪一种节目?看报纸上的电视节目预告吗?

\item[螃蟹]我不需知道每种频道的呼叫代号,而是用编码的方法在这些开关上调出我想要的虚拟情景。用行话讲,这就叫作“通过对实参数调用某一频道”。总是有大量频道在播放各种可能设想的情景。那些内容“相近似”的各频道,其呼叫代号也彼此相近似。

\item[乌龟]为什么刚才在我们第一次看虚拟的重放镜头时,你不用动那些开关呢?

\item[螃蟹]这是因为我只是把它调到一个跟事实频道很相近的频道中,跟事实相差很小。所以它只是同事实偶有不同。把它完全调到事实频道几乎是不可能的。不过这没关系,因为事实总是很乏味的。所有的重放镜头都是亊实的本来面目,这你能想象吗?多没劲!

\item[树懒]可我觉得虚拟电视这种构想整个儿就没劲。不过,要是有证据说明你这台机器可以搞出点有趣的对实节目的话,我没准儿会改变对它的看法。比方说,要是加法不满足交换律,刚才那场比赛会怎么样?

\item[螃蟹]噢,老天,噢,老天爷!我担心对于这种型号的机器来说,这个要求所需要的推理过程太复杂了。无奈我现在还没有一台在这方面很完备的超拟电视机。有了这样一台超拟电视机,就可以满足你各种花样的要求了。

\item[树懒]嗬。

\item[螃蟹]不过请看——这差不多也能做到。你们想不想瞧瞧要是$13$不是个素数,刚才那场比赛会怎么样?

\item[树懒]多谢!不必了。这毫无意义!我要是刚才那场比赛的话,一定会叫你们这样随心所欲地播来放去弄得厌烦透顶啦。哼,你们这群头脑不清的家伙在这里没完没了地改换概念!接着往下看球赛吧!

\item[阿基里斯]你从哪儿搞来的这么一套虚拟电视机,老蟹?

\item[螃蟹]信不信由你:小懒和我有天晚上去了趟乡村市场,这电视机就是我们在一次抽彩中得的头奖。一般地说,对这种无聊玩艺儿我一向不感兴趣,可当时我被某种疯狂的冲动抓住了,于是就买了一张彩票。

\item[阿基里斯]你呢,小懒?

\item[树懒]我承认我也买了一张,只不过是为了迁就迁就老蟹。

\item[螃蟹]等宣布了中奖号码,我才大吃一惊,我中彩啦!

\item[阿基里斯]不可思议!在我认识的人里,还从没有谁中过彩!

\item[螃蟹]当时我也是目瞪口呆,觉得难以置信。

\item[树懒]关于中彩的事,你还有什么别的话要讲吗,老蟹?

\item[螃蟹]没有了。哦——我那张彩票号码是$129$。他们宣布的中彩号码是$128$,只差一点。

\item[树懒]所以,你们明白了吧,他事实上根本没中,

\item[阿基里斯]不过他差一点中了……

\item[螃蟹]我更倾向于说我中了,因为我是那么接近……要是我的号码再减去个一,我就会中了。

\item[树懒]但可惜的是,老蟹,在这种情况下“差一点”和“差十万八千里”没什么两样儿。

\item[乌龟]没错。哎,你呢,小懒?你的号码是多少?

\item[树懒]我的号码是$256$——是$128$的两倍。太接近了,是吧!可我不明白那些市场官员的脑袋为什么死不开窍。他们拒绝给我完全应该归我的奖品。有个家伙声称他应该得这个奖,因为他的号码是$128$。我觉得我的号码比他的更接近,可对那些官僚们你真是一点辙都没有。

\item[阿基里斯]我整个儿一个糊涂,要是你根本就没有赢得虚拟电视机,老蟹,我们怎么能今天一下午坐在这儿看这台电视?看来我们自己好像也是呆在某个可能会存在的虚拟世界里,只要环境稍有些变化……

\item[解说员]各位观众,要是螃蟹贏得了虚拟电视机的话,这就是这个下午在螃蟹家将会发生的事情。但是因为他并没有中彩,所以这四个朋友事实上只是观看了一场本地队被打得落花流水的比赛,用这种方式度过了一个愉快的下午。本地队在这场比赛中以$0$比$128$失利——也许是$0$比$256$?哦,对于一场在五维空间中的冥王星上举行的西瓜球比赛,这是无关紧要的。

\end{dialogue}

\end{dialog}
